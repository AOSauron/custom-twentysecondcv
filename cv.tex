%%%%%%%%%%%%%%%%%%%%%%%%%%%%%%%%%%%%%%%%%
% Customized Twenty Seconds CV
%   --> French
% LaTeX Template
% Version 2.0 (03/05/18)
%
% Author: Guillaume Garcia
%
% License:
% The MIT License (see included LICENSE file)
%
%%%%%%%%%%%%%%%%%%%%%%%%%%%%%%%%%%%%%%%%%

%-----------------------------------------------------------------------------------------------------------
%                                                   TIPS
%-----------------------------------------------------------------------------------------------------------
%
%   - Pour ne pas afficher une commande/un élément il suffit de commenter la ligne correspondante avec "%"
%   - Ajouter un espacement verticale : \vspace{ymm} avec y un entier > 0 (millimètres)
%   - Réduire un espacement verticale : \vspace{-ymm} avec y un entier > 0 (millimètres)
%   - Ajouter/réduire un espacement horizontale :  \hspace{xmm} avec x un entier > 0 ou < 0 (millimètres)
%   - Italique = \textit{}
%   - Gras = \textbf{}
%   - Souligné = \underline{}
%   - Mettre de la couleur = \color{colorname} où colorname est dans twentysecondcv.cls sous COLOURS
%   - Si nécessité de changer la forme "profonde", la modifier dans le fichier twentysecondcv.cls
%
%-----------------------------------------------------------------------------------------------------------

%%%%%%%%%%%%%%%%%%%%%%%%%%%%%%%%%%%%%%%
%
%     TODO :
%       - Références [optionnel]
%       -
%
%
%%%%%%%%%%%%%%%%%%%%%%%%%%%%%%%%%%%%%%%


%%%%%%%%%%%%%%%%%%%%%%%%%%%%%%%%%%%%%%%%%%%%%%%%%%%%%%%%%%%%%%%%%%%%%%%%%%%%%%%%%%%%%%%%%%%%%%%%%%%%%%%%%%%%
%                                                                                                          %
%                           PACKAGES AND OTHER DOCUMENT CONFIGURATIONS                                     %
%                                                                                                          %
%%%%%%%%%%%%%%%%%%%%%%%%%%%%%%%%%%%%%%%%%%%%%%%%%%%%%%%%%%%%%%%%%%%%%%%%%%%%%%%%%%%%%%%%%%%%%%%%%%%%%%%%%%%%

\documentclass[letterpaper]{twentysecondcv} % a4paper for A4

\usepackage{pxfonts}                        % Pretty arrows

%-----------------------------------------------------------------------------------------------------------
%                       END OF PACKAGES AND OTHER DOCUMENT CONFIGURATIONS 
%-----------------------------------------------------------------------------------------------------------

%%%%%%%%%%%%%%%%%%%%%%%%%%%%%%%%%%%%%%%%%%%%%%%%%%%%%%%%%%%%%%%%%%%%%%%%%%%%%%%%%%%%%%%%%%%%%%%%%%%%%%%%%%%%
%                                                                                                          %
%            USER COMMANDS (partie gauche) :  Compétences / Programmation / Langues / Qualités             %
%                                                                                                          %
%%%%%%%%%%%%%%%%%%%%%%%%%%%%%%%%%%%%%%%%%%%%%%%%%%%%%%%%%%%%%%%%%%%%%%%%%%%%%%%%%%%%%%%%%%%%%%%%%%%%%%%%%%%%

%%%%%%%%%%%%%%%%%%%%%%%%%%%%%%%%%%%%%%%%%%%%%%% Compétences %%%%%%%%%%%%%%%%%%%%%%%%%%%%%%%%%%%%%%%%%%%%%%%%
\newcommand\skills{ 
~
	\smartdiagram[bubble diagram]{
        \textbf{~~Ingénierie~~}\\\textbf{~~embarquée~~},
        \textbf{~~~~IA~~~~},
        \textbf{~Logiciels~}\\\textbf{~Embarqués~}\vspace{1mm},
        \textbf{Dev}\\\textbf{Mobile},
        \textbf{Traitement}\\\textbf{du Signal},
        %\textbf{Traitement}\\\textbf{de l'image},
        \textbf{Sûreté},
        \textbf{~~~Agilité~~~}\\\textbf{~~~Scrum~~~},
        \textbf{Temps}\\\textbf{réel}\vspace{-1.5mm},
        \textbf{Électronique}
    }
} 

%%%%%%%%%%%%%%%%%%%%%%%%%%%%%%%%%%%%%%%%%%%%%% Programmation %%%%%%%%%%%%%%%%%%%%%%%%%%%%%%%%%%%%%%%%%%%%%%%
% Notation / score de la barre de progression (MAX = 6)
\newcommand\maitrise{5.5} 
\newcommand\intermediaireHaut{4.5}
\newcommand\intermediaireBas{3.5}
\programming{{Assembleur $\textbullet$ Basic$\textbullet$ VHDL / \intermediaireBas},
{C++ $\textbullet$ Matlab $\textbullet$ Qt $\textbullet$ Arduino/ \intermediaireHaut},
{Java $\textbullet$ C $\textbullet$ Python $\textbullet$ Shell $\textbullet$ \large{ \LaTeX } / \maitrise}}

%%%%%%%%%%%%%%%%%%%%%%%%%%%%%%%%%%%%%%%%%%%%%%%%% Langues %%%%%%%%%%%%%%%%%%%%%%%%%%%%%%%%%%%%%%%%%%%%%%%%%%
\newcommand\languages{
~   
    \vspace{1mm} % Inter-espace
    \textbf{\large{Anglais}}\includegraphics[scale=0.40]{img/4stars.png} $\rightsquigarrow\textit{\textbf{TOEIC} - niv. B2}$\\
    \textbf{\large{Espagnol}}\includegraphics[scale=0.40]{img/3stars.png} $\rightsquigarrow\textit{estimé niv. B1}$\\
}   

%%%%%%%%%%%%%%%%%%%%%%%%%%%%%%%%%%%%%%%%%%%%%%%%% Qualités %%%%%%%%%%%%%%%%%%%%%%%%%%%%%%%%%%%%%%%%%%%%%%%%%
\newcommand\personalskills{
\vspace{2.5mm}
$\circleddotright\textbf{ Bienveillant}$\vspace{.5mm} \hspace{1.6mm}
$\circleddotright\textbf{ Pragmatique}$\vspace{.5mm}\\
$\circleddotright\textbf{ Dynamique}$\vspace{.5mm} \hspace{2.2mm}
$\circleddotright\textbf{ Persévérant}$\vspace{.5mm}\\

%%% Format:
%$\circleddotright\textbf{ Blabla}$\vspace{.5mm}\\
}

%%%%%%%%%%%%%%%%%%%%%%%%%%%%%%%%%%%%%%%%%%%%%%% OS préférés %%%%%%%%%%%%%%%%%%%%%%%%%%%%%%%%%%%%%%%%%%%%%%%%%
%\newcommand\preferedos{ 
        %\textbf{GNU/Linux}\includegraphics[scale=0.40]{img/5stars.png}\\
        %\textbf{Windows}\includegraphics[scale=0.40]{img/4stars.png}\\
        %\textbf{MacOS}\includegraphics[scale=0.40]{img/3stars.png}\\
%}

%-----------------------------------------------------------------------------------------------------------
%                               END OF USER COMMANDS
%-----------------------------------------------------------------------------------------------------------


%%%%%%%%%%%%%%%%%%%%%%%%%%%%%%%%%%%%%%%%%%%%%%%%%%%%%%%%%%%%%%%%%%%%%%%%%%%%%%%%%%%%%%%%%%%%%%%%%%%%%%%%%%%%
%                                                                                                          %
%     INFORMATIONS PERSO : Nom Prénom / Titre / Disponibilités / Mobilité / Nationalité / Âge / Permis     %
%                                                                                                          %
%%%%%%%%%%%%%%%%%%%%%%%%%%%%%%%%%%%%%%%%%%%%%%%%%%%%%%%%%%%%%%%%%%%%%%%%%%%%%%%%%%%%%%%%%%%%%%%%%%%%%%%%%%%%

%%%%%%% Format : espacement vertical entre les éléments (mm)
\newcommand\infovspace{\vspace{-2mm}}
%%%%%%%
%%%%%%% Format : Ligne séparatrice
\newcommand\simpleline{\rule[0.25\baselineskip]{6cm}{1pt}}
%%%%%%%

% Prénom NOM
\cvname{\huge{Guillaume GARCIA}}

%%%% DÉBUT : Titre + infos diverses
\cvjobtitle{

% Titre & Recherche
\large{\color{mainblue}\textbf{Élève ingénieur du numérique}}\vspace{1mm}\\
\normalsize{\color{mainblue}{\textit{Recherche une opportunité Ingénieur \\Développement Logiciel Embarqué}}}\\
\simpleline

% Disponibilités
\infovspace$\Diamonddotright$
\textnormal{\normalsize{ Disponible dès le \textbf{
                                                        01/10/2018
                                                }}}

% Mobilité
\infovspace
$\Diamonddotright$\normalsize{ \textbf{ Mobilité} :\newline
\textit{France-Espagne-Lux-Belgique-Suisse}}\\
\simpleline

% Nationalité
\infovspace$\Diamonddotright$ 
\normalsize{\textit{ Nationalité Française}

% Âge
\infovspace$\Diamonddotright$
\textit{ 24 ans}

% Permis
\infovspace$\Diamonddotright$ 
\textit{ Permis B - Véhiculé}}\\
\simpleline
\infovspace

} %%%% FIN : Titre + infos diverses

% eMail
\infovspace
\cvmail{guillaume.garcia@telecomnancy.net} 

% LinkedIn
\infovspace
\cvlinkedin{/in/garcia-guillaume}

% GitHub
\infovspace
\cvgithub{AOSauron}

% Numéro de Téléphone
\infovspace
\cvnumberphone{(+33) 6 03 99 10 66}

% Adresse OU site internet perso
\infovspace
\cvsite{95320 Saint-Leu-La-Forêt - France}


%-----------------------------------------------------------------------------------------------------------
%                                  END OF PERSONNAL INFORMATIONS
%-----------------------------------------------------------------------------------------------------------

\begin{document}

\makeprofile % Print the leftsidebar

\vspace{-0.40cm} % Mise en forme

%%%%%%%%%%%%%%%%%%%%%%%%%%%%%%%%%%%%%%%%%%%%%%%%%%%%%%%%%%%%%%%%%%%%%%%%%%%%%%%%%%%%%%%%%%%%%%%%%%%%%%%%%%%%
%                                                                                                          %
%                                           FORMATION                                                      %
%                                                                                                          %
%%%%%%%%%%%%%%%%%%%%%%%%%%%%%%%%%%%%%%%%%%%%%%%%%%%%%%%%%%%%%%%%%%%%%%%%%%%%%%%%%%%%%%%%%%%%%%%%%%%%%%%%%%%%
\section{Formation}
\vspace{-0.1cm}
\begin{twenty}  

    % FORMAT POUR UNE FORMATION : \twentyitem{<dates>}{<title>}{<organization>}{<location>}{<description>}
    
	\twentyitem
    	{2015 - 2018}
        {}
        {Cycle ingénieur informatique \textnormal{ \textit{Logiciel Embarqué}}\\\textnormal{ \textit{ Diplôme attendu fin 2018}}}
        {\\\textbf{TELECOM Nancy}, Université de Lorraine, Villers-lès-Nancy 54600, Lorraine}
        {}
        {}
	\twentyitem
    	{2014 - 2015}
		{}
        {Classe préparatoire MP 5/2 \textnormal{\textit{ Maths \& Physique}}}
        {\\Lycée Jacques Decour, Paris 75009, Île-de-France}
        {}
        {}
	\twentyitem
    	{2012 - 2014}
		{}
        {Classes préparatoires MPSI puis MP \textnormal{\textit{ Maths \& Physique}}}
        {\\Lycée Gustave Monod, Enghien-les-bains 95880, Île-de-France}
        {}
        {}
	\twentyitem
    	{2009 - 2012}
		{}
        {Baccalauréat S option SVT \textnormal{ \textit{spé physique-chimie, Mention Bien}}}
        {\\Lycée Notre Dame de Bury, Margency 95580, France}
        {}
        {}
\end{twenty}

%-----------------------------------------------------------------------------------------------------------
%                                       END OF FORMATION
%-----------------------------------------------------------------------------------------------------------

\vspace{-0.55cm} % Mise en forme

%%%%%%%%%%%%%%%%%%%%%%%%%%%%%%%%%%%%%%%%%%%%%%%%%%%%%%%%%%%%%%%%%%%%%%%%%%%%%%%%%%%%%%%%%%%%%%%%%%%%%%%%%%%%
%                                                                                                          %
%                                 EXPÉRIENCES PROFESSIONNELLES                                             %
%                                                                                                          %
%%%%%%%%%%%%%%%%%%%%%%%%%%%%%%%%%%%%%%%%%%%%%%%%%%%%%%%%%%%%%%%%%%%%%%%%%%%%%%%%%%%%%%%%%%%%%%%%%%%%%%%%%%%%
\section{Expériences professionnelles}
\vspace{-0.25cm}

\begin{twenty}

    % FORMAT POUR UNE EXPERIENCE : \twentyitem{<dates>}{<title>}{<location>}{<description>}
    
    \twentyitem
    	{Avril 2018}
		{$\DiamonddotRight$\textit{\small{ 6 mois}}}
        {Stage ingénieur informatique}
        {\href{https://www.thalesgroup.com/fr/worldwide/defense/practical-information}{\includegraphics[scale=0.35]{img/thales-logo.png} \hspace{0.5cm} (92230)}}
        {}
        {\vspace{-0.20cm}
            \begin{itemize}
                \item Spécifier, développer et qualifier un \textit{simulateur de trajectoires}.% dans le cadre du développement d'un radar secondaire nouvelle génération (à panneaux fixes).
                \item Développement AGILE en deux parties (\textbf{moteur} \& \textbf{IHM}) du simulateur dont le but est de tester les algorithmes de pistages d'un radar secondaire nouvelle génération. (IHM \textbf{Python 2}, moteur en \textbf{C}, \textbf{AGILE/Scrum})
            \end{itemize}
            \vspace{-0.20cm}
        }
        \\
    \twentyitem
    	{Juin 2017}
		{$\Diamonddotright$\textit{\small{ 2 mois}}}
        {Stage technicien informatique}
        {\href{https://www.thalesgroup.com/fr/worldwide/defense/practical-information}{\includegraphics[scale=0.35]{img/thales-logo.png} \hspace{0.5cm} (92230)}}
        {}
        {\vspace{-0.20cm}
            \begin{itemize}
                \item Projet \textit{Code Beautifier} : Outil de vérification et formatage de règles de codage, fourni avec 2 crochets \textbf{git}. (\textbf{Python 2}, \textbf{JSON} et \textbf{YAML})
            \end{itemize}
            \vspace{-0.20cm}
        }
        \\
    %\vspace{-0.1cm}
	\twentyitem
    	{Juil 2016}
    	{$\Diamonddotright$\textit{\small{ 1 mois}}}
        {Employé des services techniques}
        {\href{http://www.saint-leu-la-foret.fr/}{\includegraphics[scale=0.18]{img/saint-leu-la-foret.png} \hspace{0.5cm} (95320)}}
        {}
        {\vspace{-0.20cm}
            {\begin{itemize}
                \item Plomberie simple, peinture en bâtiment, déménagement de matériel, peinture routière, réparation diverses, en équipe de 3.
            \end{itemize}}
            \vspace{-0.20cm}
        }
    \\   
    \twentyitem
   		{Juil 2014}
		{$\Diamonddotright$\textit{\small{ 2 mois}}}
        {Employé commercial}
        {\href{http://www.cora.fr/ermont/}{\includegraphics[scale=0.05]{img/Logo_cora_2011.png} \hspace{0.5cm}(95120)}}
        {}
        {\vspace{-0.20cm}
            \begin{itemize}
                \item Manutention de produits commerciaux, \textit{facing} des rayons d'hypermarché.
            \end{itemize}
            %\vspace{-0.10cm}
        }
\end{twenty}\\

%-----------------------------------------------------------------------------------------------------------
%                               END OF EXPÉRIENCES PROFESSIONNELLES
%-----------------------------------------------------------------------------------------------------------

\vspace{-0.1cm} % Mise en forme

%%%%%%%%%%%%%%%%%%%%%%%%%%%%%%%%%%%%%%%%%%%%%%%%%%%%%%%%%%%%%%%%%%%%%%%%%%%%%%%%%%%%%%%%%%%%%%%%%%%%%%%%%%%%
%                                                                                                          %
%                                    PROJETS PERSO & SCOLAIRES                                             %
%                                                                                                          %
%%%%%%%%%%%%%%%%%%%%%%%%%%%%%%%%%%%%%%%%%%%%%%%%%%%%%%%%%%%%%%%%%%%%%%%%%%%%%%%%%%%%%%%%%%%%%%%%%%%%%%%%%%%%


\section{Projets scolaires et personnels}
\vspace{-0.15cm}

\begin{itemize}

    % FORMAT USUEL POUR PROJET : \textbf{projectname} - explanation(s)
    
    \item \textbf{Linux embarqué} $\rightsquigarrow$ {Développement de \textbf{drivers} pour sonar de proximité et afficheur 7-segments à 3 digits pour un Debian embarqué sur une carte iMX233-OLinuXin-MAXI. (\textbf{ARM, GPIOs, C Noyau, Electronique embarquée})}
    \item \textbf{Projet Industriel 2017} $\rightsquigarrow$ \textit{(\textbf{Chef de Projet} - projet sur 6 mois)} Boîtier connecté au bus CAN d’une voiture afin d’en effectuer le diagnostic en cas de panne (et envoyer ce rapport par GSM), pour le compte de la société \textit{Clarion}. (\textbf{C, Python, Réseaux, Électronique embarquée, ODBII})
    \item \textbf{ptar} $\rightsquigarrow$ Extracteur d'archives tar et tar.gz pour les systèmes UNIX, durable et multithreadé. (\textbf{C, multi-threading})
    \item \textbf{clooc} $\rightsquigarrow$ Compilateur produisant du langage d'assemblage. Il compile un langage orienté objet simple en assembleur simplifié. (\textbf{Java, Antlr})
    \item \textbf{noise-sonar} $\rightsquigarrow$ Simulation d'un sonar à bruit. (\textbf{Matlab})
    
    % Vieux projets / moins intéressant :
    
    %\item \textbf{myAdblock} – Serveur proxy bloqueur de publicités, écrit en \textit{C}.
    %\item \textbf{PIDR} - Remontée de données par boîtier Arduino relié au bus CAN d'une voiture électrique sur un serveur distant. (\textbf{C++ Arduino, Python})
    %\item \textbf{aOS} - Projet \textit{personnel} de création d'un système Linux en suivant le livre Linux From Scratch. (\textbf{Shell, systèmes linux})
    
\end{itemize}

%-----------------------------------------------------------------------------------------------------------
%                                END OF PROJETS PERSO & SCOLAIRES 
%-----------------------------------------------------------------------------------------------------------

%%%%%%%%%%%%%%%%%%%%%%%%%%%%%%%%%%%%%%%%%%%%%%%%%%%%%%%%%%%%%%%%%%%%%%%%%%%%%%%%%%%%%%%%%%%%%%%%%%%%%%%%%%%%
%                                                                                                          %
%                                           CENTRES D'INTÉRÊT                                              %
%                                                                                                          %
%%%%%%%%%%%%%%%%%%%%%%%%%%%%%%%%%%%%%%%%%%%%%%%%%%%%%%%%%%%%%%%%%%%%%%%%%%%%%%%%%%%%%%%%%%%%%%%%%%%%%%%%%%%%

\section{Centres d'intérêt}
\vspace{-0.15cm}

\begin{itemize}

    \item Passionné par les \textbf{sciences}, en particulier l'\textit{astronomie}, la \textit{physique}, la \textit{biologie} et l'\textit{informatique}. Grand amateur de \textbf{mécanique automobile} (pratiquée sur mon propre véhicule).
    
    \item Apprécie les sports d'endurance notamment le {\textbf{VTT}} (pratique amateur intense en duo). Pratique du {\textbf{modélisme navale}} (4 ans en club) et de la {\textbf{pêche} en eau douce}. 
    
    \item Séjours réguliers en \textbf{Andalousie, Espagne} (\textit{Almería}).
    
    \item Amateur de \textbf{cinéma} et de \textbf{jeux vidéos} (\textit{stratégie}, \textit{RPG}) orientés \textit{science-fiction}, \textit{action}, \textit{fantastique}.
    
\end{itemize}

%-----------------------------------------------------------------------------------------------------------
%                                       END OF CENTRES D'INTÉRÊT
%-----------------------------------------------------------------------------------------------------------

\end{document} 
